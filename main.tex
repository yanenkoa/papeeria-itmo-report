% Шаблон отчёта по лабораторным работам (Версия от 15.02.2016)
% предназначен для использования студентами каф. ПМиИ СамГТУ
% при оформлении отчетов по лабораторным работам. 
% Для настройки пакета listigs использовался материал 
% статьи Михаила Конника aka virens
% <http://mydebianblog.blogspot.ru/2012/12/latex.html>
% Copyright (c) 2016 by Mikhail Saushkin (msaushkin@gmail.com) 
% All rights reserved except the rights granted by the
% Creative Commons Attribution 4.0 International
% <https://creativecommons.org/licenses/by/4.0/>
% Свежая версия шаблона здесь <https://www.overleaf.com/read/sqvxbnhgxxdm>
\documentclass[12pt]{report}
\usepackage[a4paper, mag=1000, left=3cm, right=1.5cm, top=0cm, bottom=2cm, headheight=1.82cm, headsep=0.15cm, includehead, includefoot]{geometry}
% \usepackage[T1]{fontenc}
% \usepackage[libertine]{newtxtext}
% \usepackage{librebaskerville}
\usepackage[utf8]{inputenc}
\usepackage[english,russian]{babel}
\usepackage{tempora}
\usepackage{newtxmath}
\usepackage{titlesec} 
\usepackage{caption}
\usepackage{fancyhdr}
\usepackage{indentfirst}


% \pagestyle{myheadings}
% \setlength{\headerskip}{3cm}

% \lhead{}
% \chead{}
% \rhead{}
% \lfoot{}
% \cfoot{}
% \rfoot{}
\fancyhf{}
\chead{
% {\setstretch{1.5}
\thepage
% }
}
\renewcommand{\headrulewidth}{0pt}

\pagestyle{fancy}

\renewcommand{\baselinestretch}{1.5}

\setlength{\parindent}{1.25cm}

\def\thesection{\arabic{section}}
\titleformat{\section}{\large\center\bf}{\thesection.}{0.3cm}{\MakeUppercase}
\titlespacing*{\chapter}{0.63cm}{24pt}{12pt}

\begin{document}
% Переоформление некоторых стандартных названий
\def\contentsname{Содержание}

% Оформление титульного листа
\begin{titlepage}

\newgeometry{a4paper, mag=1000, left=3cm, right=1.5cm, top=0cm, bottom=2.2cm, headheight=1.82cm, headsep=0cm, includehead, ignorefoot}
\begin{center}
\textsc{МИНИСТЕРСТВО ОБРАЗОВАНИЯ И НАУКИ РОССИЙСКОЙ ФЕДЕРАЦИИ\\
УНИВЕРСИТЕТ ИТМО\\
Кафедра высокопроизводительных вычислений
}

\vspace{6.68cm}

\textbf{\large ОТЧЁТ}\\[4.5mm]
О выполнении лабораторной работы № 4\\
«Симулирование $n$ тел с помощью графического процессора и технологии CUDA»\\[18mm]
\end{center}
% \vspace{0.1cm}
\hspace{\fill}
\begin{minipage}{.48\textwidth}
\textbf{Работу выполнил:}\\
ст. группы М4117 Яненко А.С.\\
\textbf{Работу принял:}\\
\hspace*{3.25cm} Болгова Е.В.\\
\end{minipage}
\vfill
\begin{center}
Санкт-Петербург\\
2017
\end{center}
\end{titlepage}

\restoregeometry

% Содержание
% \tableofcontents

\setcounter{page}{2}

\section{ЦЕЛЬ РАБОТЫ}
Какая цель преследуется в рамках выполнения лабораторной работы (2-3 строки).

\section{ПОСТАНОВКА ЗАДАЧИ}
Конкретная задача, которая решается в рамках выполнения лабораторной работы (1 абзац на 0,2 – 0,3 стр.).
\section{КРАТКАЯ ТЕОРЕТИЧЕСКАЯ ЧАСТЬ}
Краткие сведения о теме дисциплины, в рамках которого выполняется данная лабораторная работа. Сведения о используемых методах, методиках, алгоритмах: достоинства, свойства, недостатки (максимум 1 стр.). Своими словами, а не copy-past!
\section{РЕЗУЛЬТАТ}
Представление результатов (промежуточные и итоговые изображения). Краткое обсуждение результатов (что означает конкретные значения результатов…) – 1 – 3 стр.
\section{ЗАКЛЮЧЕНИЕ}
Что сделано. Какие навыки и умения были получены. Прогноз возможностей применения навыков и умений, а также полученных результатов (5 – 10 строк).


\end{document}
